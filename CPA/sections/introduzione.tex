\chapter{Introduzione}
\textbf{Crittologia}: l'arte delle scritture segrete, può essere divisa in 3 macro argomenti:
\begin{enumerate}[nosep]
    \item \textbf{crittografia}: come \textit{trasformare} messaggi per proteggerne il contenuto (l'informazione).
    \item \textbf{steganografia}: come \textit{nascondere} messaggi per evitare che venga individuato (ex. \textbf{least significant bit steganography}).
    \item \textbf{crittoanalisi}: come \textit{analizzare} messaggi e rivelarne l'informazione.   
\end{enumerate}

\section{Crittografia Classica}
La sicurezza della crittografia classica si basa unicamente sulla \textbf{segretezza} del \textbf{metodo} (noto solo al \textit{sender} e al \textit{receiver}), considerava come una tipologia di attacco quello \textbf{passivo} (\textit{read only}). Basandosi su questi concetti il suo utilizzo in applicazioni reali è molto limitato (nel senso moderno), considerando la comunicazione in termini di scambio di informazioni in linguaggio naturale. \\
Alcuni esempi: \textbf{scytale} (\textit{transposition cipher}), \textbf{caesar cipher} (\textit{shift cipher}) e \textbf{vigenere cipher}.

\newpage
\section{Crittografia Moderna}
\subsection{Encryption only}
La crittografia moderna si basa su due principi:
\begin{enumerate}[nosep]
    \item \textbf{Kerckhoffs principles}:
    \begin{itemize}[nosep]
        \item gli algoritmi devono essere pubblici.
        \item la sicurezza del metodo si deve basare sulla \textbf{segretezza} della \textbf{chiave}.
        \item uno schema deve essere ``praticamente'', se non ``matematicamente'' indecifrabile
    \end{itemize}
    \item \textbf{Shannon principles}:
    \begin{itemize}[nosep]
        \item \textbf{confusione}: ogni bit del crittogramma deve dipendere da più bit della chiave, oscurando, però, la correlazione tra le due.
        \item \textbf{diffusione}: se viene cambiato un singolo bit del testo in chiaro, allora almeno la metà dei bit del crittogramma devono cambiare, e viceversa.
    \end{itemize}
\end{enumerate}
Nella crittografia moderna lo spazio delle chiavi deve essere sufficentemente ampio per evitare una ricerca esaustiva su di esso, in più, nessuna informazione (nè del \textit{plaintext}, né della \textit{key}) deve poter essere estrapolata dal \textit{ciphertext}. Viene detto che il \textit{ciphertext} deve essere \textbf{indistinguibile} da una sequenza di bit \textit{random}.

\subsection{Extended and applied settings}
Gli avversari (i crittoanalisti) non sono più unicamente \textbf{passivi}, ma bisogna modellare delle tipologia di avversari che siano capaci anche di \textbf{interagire} con i nostri sistemi e \textbf{manipolare} dei messaggi. \\
Per ognuna di queste modellazioni è necessario \textbf{provare la sicurezza} dei sistemi andando a \textbf{definire} delle attività (tramite la comprensione e modellare cosa è ``sicuro'') e \textbf{costruendo} attività (progettandole e provandone la veridicità). \\ \newline
Bisogna definire le \textbf{primitive}, gli \textbf{schemi}, i \textbf{protocolli} e le \textit{applicazioni} che vengono utilizzati, andandoli ad analizzare separatamente e completa. \\
$\Rightarrow$ può essere necessario costruire schemi e protocolli modellati su misura per applicazioni reali inerenti ad un certo caso d'uso: \textcolor{red}{\textbf{Applied Cryptography}}.

\newpage
\section{Crittografia Applicata}
È un layer di astrazione che può essere (quasi) direttamente mappato all'interno di una soluzione per un caso d'uso reale (quindi tecniche ``pratiche''). Siccome analizziamo \textbf{soluzioni pratiche} bisogna gestire possibili errori dovuti ad \textbf{implementazioni} o \textbf{\textit{deployment}} errati. I protcolli sicuri assumono che un attaccante tenti di accedere alle informazioni in transito (violazione della \textbf{confidenzialità}) e cerchi di impersonificare un mittente (violazione dell'\textbf{autenticazione}). \\ 
Una delle sicurezze che deve garantire un protocollo sicuro è la \textbf{confidenzialità}. \\ \newline
Quando si studi/analizza/progetta un protocollo crittografico è necessario identificare:
\begin{itemize}[nosep]
    \item \textbf{system model}: descrive lo scenario (``idealmente'') di utilizzo, andando a definire: gli attori \textbf{legittimi}, la \textbf{tipologia di protocollo} utilizzata, le \textbf{informazioni} possedute dagli attori legittimi, e altre informazioni sullo scenario applicativo.
    \begin{boxA}
        \textcolor{orange}{\textbf{Esempio}} \\
        Un protocollo sicuro che ha come \textbf{scopo} proteggere le informazioni scambiati tra due attori: \textbf{Alice} e \textbf{Bob} quando un possibile attaccante \textbf{Eve} può accedere direttamente all'informazione tramite il canale fisico. 
        \smallskip

        {\centering
            \includegraphics[width=0.3\textwidth]{img/system_model.png}
        \par}
    \end{boxA}
    \item \textbf{threat modelling} (modellazione della tipologia di attaccante): abbiamo identificato degli attori legittimi, ma quanto sono affidabili? Modelliamo il protocollo sulla base dell'attaccante.
    \begin{enumerate}[nosep]
        \item che operazioni può effettuare sui dati: solo lettura, modifica, inserimento o eliminazione dei dati.
        \item qual è la superficie di attacco e cosa può provare a fare: ha accesso ad alcune funzionalità (cifrazione/decifrazione), che tipologia conoscenza (\textit{white}/\textit{gray}/\textit{black box}), quanti tentavi si hanno: adattivo o meno.
    \end{enumerate}
    \begin{boxA}
        \textcolor{orange}{\textbf{Esempio}} \\
        Cosa può fare \textbf{Eve} per compromettere la comunicazione: leggere, manipolare le informazioni in transito, compromissione di un attore legittimo.
        \smallskip

        {\centering
            \includegraphics[width=0.3\textwidth]{img/threat_modelling.png}
        \par}
    \end{boxA}
    \item \textbf{security guarantees}: quale aspetto di sicurezza vogliamo garantire: \textbf{confidenzialità}, \textbf{integrità} (autenticazione), \textbf{disponibilità}, \textbf{non ripudio} (è anche presente il concetto di \textbf{\textit{forward security}}).
    \item \textbf{cryptogrphy settings}: le due classi principali sono: crittografia \textbf{simmetrica} (le funzioni di \textit{encrypt} e \textit{decrypt} utilizzano lo stesso \textbf{segreto}) e crittografia \textbf{asimmetrica} (sono presenti due differenti \textbf{chiavi}, uno utilizzabile durante la funzione di \textit{encrypt} - \textit{public} - e l'altro utilizzato durante la funzione di \textit{decrypt} - \textit{secret}).
    \item \textbf{security assumptions of a proposed scheme}
\end{itemize}

Alcuni \textbf{protocolli}:
\begin{enumerate}[nosep]
    \item \textbf{\textit{Secure key exchange protocol}} (scambio sicuro di chiavi): Alice e Bob non hanno nessuna \textbf{chiave}, ne vogliono ottenere una \textbf{sicura} e \textbf{condivisa} comunicando su un canale sincrono e non sicuro.
    \item \textbf{\textit{Secure storage}}: gli algoritmi di crittografia possono aggiungere protezione su dati conservati in canali protetti, l'avversario ha avuto accesso ai dati dopo aver sconfitto le difese iniziali, negli scenari di storage possiamo andare ad analizzare due tipologie di dati: ``\textit{data at rest}'' (è lo standard e la \textit{best-practice}) oppure ``\textit{data in use}'' (continua ricerca soprattuto per i dispositivi mobili).
    \item \textbf{\textit{(Identity) Authentication}}: (\textbf{\textit{challenge-response protocols}}) ovvero dimostrare il possesso di un segreto senza mandarlo.
    \item \textbf{\textit{Password Protection}}: gli schemi crittografici possono ``proteggere'' le password in caso di \textit{data breaches}, non solo usando l'hash (anche se aggiunto il \textit{salt}), ma anche tecniche per prevenire il \textbf{\textit{brute-forcing}} attraverso \textbf{ASICs} (\textbf{Application-specific integrated circuit}).
\end{enumerate}
Per riassumere quanto visto fino ad adesso, possiamo dire che la \textbf{crittografia applicata} è l'insieme di molti \textbf{\textit{system models}} (comunicazioni sincroni, messaggi di gruppi asincroni, \textit{disk encryption}, ...), molti \textbf{\textit{security guarantees}} (\textit{information security}), integrità e autenticazione, molti \textbf{\textbf{type of attackers}} (passivi e attivi oppure online e offline). La \textbf{crittografia applicata} mira a dimostrare la sicurezza sfruttando ``strati inferiori'':
\begin{itemize}[nosep]
    \item la sicurezza dei \textbf{protocolli} viene ridotta alla sicurezza sugli schemi.
    \item la sicurezza degli \textbf{schemi} viene ridotta alla sicurezza delle primitive.
    \item la sicurezza delle \textbf{primitive} viene ridotta alla robusteza di problemi matematici.
\end{itemize}
Per progettare un \textbf{protocollo di comunicazione sicuro} è necessario utilizzare un approccio modulare basato su uno \textit{stack} di altri ``oggetti'', un approccio tipico per lo \textit{stack} è quello che ogni \textit{layer} ha un determinato compito: il protocollo o lo schema è modificato su certi \textbf{\textit{cryptographic settings}} e la sicurezza viene provato contro uno specifico tipo di avversario (\textbf{\textit{security models}}).