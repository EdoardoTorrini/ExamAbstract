\chapter{Introduzione}

\section{Funzioni}
Una \textbf{funzione} o \textbf{applicazione} tra due insiemi A e B è rappresentata:
\begin{center}
    $f : A \rightarrow B \; t.c. \; \forall a \in A \; \exists ! b \in B \; | \; f(a) = b$
\end{center}
\begin{enumerate}
    \item la funzione si dice \textbf{iniettiva} se:
        \begin{center}
            $\forall a, a' \in A, \; f(a) = f(a') \Rightarrow a = a'$
        \end{center}
    \item la funzione si dice \textbf{suriettiva} se:
        \begin{center}
            $\forall b \in B, \exists a \in A \; | \; f(a) = b$    
        \end{center}
    \item una funzione $f : A \rightarrow B$ si dice \textbf{biettiva} o \textbf{biunivoca} se è contemporaneamente \textit{iniettiva} e \textit{suriettiva} ovvero se:
        \begin{center}
            $\forall b \in B \; \exists ! a \in A \; t.c. \; f(a) = b $
        \end{center}
\end{enumerate}

\newpage
\section{Insiemi Discreti}
Due insiemi A e B si dicono \textbf{equipotenti} (o con la stessa \textbf{cardinalità}) se:
\begin{center}
    $f : A \rightarrow B, \; f \; biunivoca$
\end{center}
E utilizzeremo come notazione: $card(A) = card(B)$, $|A| = |B|$ oppure $\#A = \#B$.
Un insieme A si dice finito se:
\begin{center}
    $\exists n \in \mathbb{N}, \; f : A \rightarrow \mathbb{N}_n, \; f \; biunivoca$
\end{center}
In questo caso diremo che la \textbf{cardinalità} di A è \textbf{n}: $card(A) = card(\mathbb{N}_n) = n$ \\
Un insieme A si dice \textbf{numerabile} se:
\begin{center}
    $\exists f : A \rightarrow \mathbb{N}, \; f \; biunivoca$
\end{center}
In questo caso si dice che A ha cardinalità numerabile e si può rappresentare attraverso la lettera \textbf{aleph} (è la prima lettera dell'alfabeto ebraico): $card(A) = card(\mathbb{N}) = \aleph_0$. \\
Alcuni esempi: 
\begin{enumerate}
    \item l'insieme $\mathbb{Z}$ è \textbf{numerabile} ($\#\mathbb{N} = \#\mathbb{Z}$):
        \begin{center}
            $0 \rightarrow 1$ \\
            $1 \rightarrow 2 \;\;\;\; -1 \rightarrow 3$ \\
            $2 \rightarrow 4 \;\;\;\; -2 \rightarrow 5$
        \end{center}
    possiamo quindi mappare i valori \textbf{positivi} dell'insieme $\mathbb{Z}$ sono mappati nei valori \textbf{pari} dell'insieme $\mathbb{N}$ e in maniera complementare i valori \textbf{negativi} dell'insieme $\mathbb{Z}$ sono mappati nei valori \textbf{dispari} dell'insieme $\mathbb{N}$. È quindi possibile verificare la biunivocità dell'applicazione che mappa i valori da $\mathbb{Z}$ a $\mathbb{N}$.
    \item l'insieme dei numeri \textbf{pari} $\mathbb{P}$ può definirsi numerabile, infatti: $\#\mathbb{P} = \#\mathbb{N}$, in questo caso avremo l'applicazione biunivoca del tipo:
        \begin{center}
            $f : \mathbb{P} \rightarrow \mathbb{N} \; | \; \forall p = 2n \in \mathbb{P}, \; f(p) = \frac{1}{2}p = n$
        \end{center}
\end{enumerate}

Un insieme A si dice \textbf{discreto} se è \textbf{finito} o \textbf{numerabile}. \\
Se A è finito di cardinalità \textbf{n}, i suoi elementi possono essere etichettati con gli elementi di $\mathbb{N}_n$:
\begin{center}
    $A = \{a_1, a_2, ..., a_n\}$    
\end{center}
Se A è numerabile, gli elementi possono essere ``etichettati'' con gli elementi di $\mathbb{N}$:
\begin{center}
    $A = \{a_1, a_2, ..., a_n, ...\} = \{a_i \; | \; i \in \mathbb{N} \}$
\end{center}

Dato un insieme discreto A ed un suo sottoinsieme $Y \subseteq A$ si dice \textbf{funzione caratteristica} di Y la funzione:
\begin{figure}[h]
    \begin{minipage}{0.5\textwidth}
        \centering
        $f_Y : A \rightarrow \{0, 1\} \; \forall a \in A$
    \end{minipage}
    \begin{minipage}{0.5\textwidth}
        \centering
        \begin{math}
            f_Y(a) = 
            \begin{cases}
                1 \;\;\;\; se \; a \in A \\
                0 \;\;\;\; se \; a \notin A
            \end{cases}
        \end{math}
    \end{minipage}
\end{figure} \\
Nel caso in cui A sia un insieme finito avremo che: $\#A = \sum_{a \in A}{f_Y(a)}$. \\
Se A è un insieme discreto, ed $f : A \rightarrow \{0, 1\}$ una applicazione a valori in $\{0, 1\}$, risulta univocamente determinato il sottoinsieme $Y \subseteq A$ tale che $f$ sia una funzione caratteristica di $Y$:
\begin{center}
    $Y = \{a \in A \; | \; f(a) = 1\}$
\end{center}
Un esempio, definiamo $A = \mathbb{N}$ e sia $f : A \rightarrow \{0, 1\}$ definita da una \textbf{funzione caratteristica} del tipo: $n \rightarrow \frac{1 + (-1)^{n}}{2}$. In questo caso la funzione $f$ identifica, a partire dall'insieme $\mathbb{N}$, il sottoinsieme $\mathbb{P}$ dei numeri pari. \\
Utilizzando la \textbf{funzione caratteristica} si può ricavare la seguente proprietà degli insiemi discreti:
\begin{itemize}
    \item se A è finito di cardinalità \textit{n}, l'insieme $\mathcal{P}(a)$ delle \textbf{parti di A} è in corrispondenza biunivoca con l'\textbf{insieme delle n-ple} a valori in $\{0, 1\}$.
    \item se A è numerabile, l'insieme $\mathcal{P}(a)$ delle parti di A è in corrispondenza biunivoca con l'\textbf{insieme delle successioni} a valori in $\{0, 1\}$.
\end{itemize}

\newpage
\subsection{Proprietà 1}
Se X e Y sono insiemi \textbf{finiti}, con $\#X = n, \; \#Y = m$ e con $X \cap Y = \emptyset$, allora $\#(X \cup Y) = n + m$. \\
\textcolor{green}{\textbf{Dimostrazione}}: per Hp. esistono due funzioni biettive $f : X \rightarrow \mathbb{N}_n$ e $g : Y \rightarrow \mathbb{N}_m$. Per dimostrare la proprietà occorre costruire una funzione biettiva $h : X \cup Y \rightarrow \mathbb{N}_{n + m}$. Possiamo porre $\forall c \in X \cup Y$ come:
\begin{center}
    \begin{math}
        h(c) = 
        \begin{cases}
            f(c) \qquad \;\; se \; c \in X \\
            g(c) + n \; \;\; se \; c \in Y
        \end{cases}
    \end{math}
\end{center}

\subsection{Proprietà 2}
Se X è un insieme \textbf{finito} con $\#X = n$ ed Y è un insieme \textbf{numerabile}, con $X \cap Y = \emptyset$ allora $\#(X \cup Y)$ è \textbf{numerabile}. \\
\textcolor{green}{\textbf{Dimostrazione}}: per Hp. esistono due funzioni biettive $f : X \rightarrow \mathbb{N}_n$ e $g : Y \rightarrow \mathbb{N}$. Per dimostrare la proprietà occorre costruire una funzione biettiva $h : X \cup Y \rightarrow \mathbb{N}$. Possiamo porre $\forall c \in X \cup Y$ come:
\begin{center}
    \begin{math}
        h(c) = 
        \begin{cases}
            f(c) \qquad \;\; se \; c \in X \\
            g(c) + n \; \;\; se \; c \in Y
        \end{cases}
    \end{math}
\end{center}

\fbox{\begin{minipage}{\textwidth}
    \textcolor{purple}{\textbf{Off-Topic}}: \\
    \textbf{Paradosso del Grand Hotel di Hilbert}: il paradosso del \textit{Grand Hotel} inventato dal matematico \textit{David Hilbert} per mostrare alcune caratteristiche del concetto di infinito e le differenze fra opzioni con insieme finiti ed infiniti. Hilbert immagina un hotel con infinite stanze, tutte occupate, e afferma che qualsiasi sia il numero di altri ospiti che sopraggiungano, sarà sempre possibile ospitarli tutti, anche se il loro numero è infinito, purché numerabile. \\
    Nel caso semplice, arriva un singolo nuovo ospite. Il furbo albergatore sposterà tutti i clienti nella camera successiva (l'ospite della 1 alla 2, quello della 2 alla 3, etc.); in questo modo, benché l'albergo fosse pieno è comunque, essendo infinito, possibile sistemare il nuovo ospite.
\end{minipage}}

\newpage
\subsection{Proprietà 3}
Se X e Y sono due insiemi \textbf{numerabili}, allora anche $X \cup Y$ è \textbf{numerabile}. \\
\textcolor{green}{\textbf{Dimostrazione}}: senza perdere di generalità, supponiamo che $X \cap Y = \emptyset$. Per ipotesi esistono due funzioni biettive $f : X \rightarrow \mathbb{N}$ e $g : Y \rightarrow \mathbb{N}$. Per dimostrare la proprietà occorre costruire una funzione biettiva $h : X \cup Y \rightarrow \mathbb{N}$. Ad esempio, $\forall c \in (X \cup Y)$, si può porre:
\begin{center}
    \begin{math}
        h(c) = 
        \begin{cases}
            2f(c) - 1 \quad se \; c \in X \\
            2g(c) \;\;\; \qquad se \; c \in Y
        \end{cases}
    \end{math}
\end{center}

\fbox{\begin{minipage}{\textwidth}
    \textcolor{purple}{\textbf{Off-Topic}}: \\
    \textbf{Paradosso del Grand Hotel di Hilbert}: Un caso meno intuitivo si ha quando arrivano infiniti nuovi ospiti. Sarebbe possibile procedere nel modo visto in precedenza, ma solo scomodando infinite volte gli ospiti (già spazientiti dal precedente spostamento): sostiene allora Hilbert che la soluzione sta semplicemente nello spostare ogni ospite nella stanza con numero doppio rispetto a quello attuale (dalla 1 alla 2, dalla 2 alla 4,etc.), lasciando ai nuovi ospiti tutte le camere con i numeri dispari, che sono essi stessi infiniti, risolvendo dunque il problema. Gli ospiti sono tutti dunque sistemati, benché l'albergo fosse pieno.
\end{minipage}}
\\ \newline
\textbf{Proposizione}: se X è un insieme numerabile e $Y \subseteq X$ allora Y è un insieme \textbf{discreto}.

\newpage
\subsection{Proprietà 4}
Se $\{A_i \; | \; i \in \mathbb{N}\} = \{A_1, A_2, ..., A_i, ...\}$ è un \textbf{insieme numerabile} di \textbf{insiemi numerabili}, si ha che:
\begin{center}
    $\#(\bigcup_{i \in \mathbb{N}}A_i) = \#\mathbb{N}$
\end{center}
\textcolor{green}{\textbf{Dimostrazione}}: senza perdere di generalità, supponiamo che gli insiemi siano fra loro \textbf{disgiunto}: $A_i \cap A_j = \emptyset, \; \forall i \in j$. Per dimostrare la tesi, utilizziamo il \textit{procedimento diagonale di \textbf{Cantor}}, enumerando per righe gli elementi di ciascun insieme:

\begin{center}
    \begin{tabular}{ccccccc}
        $A_1$: & \textcolor{red}{$a_{11}$} & \textcolor{blue}{$a_{12}$} & \textcolor{purple}{$a_{13}$} & \textcolor{brown}{...} & \textcolor{green}{$a_{1h}$} & ... \\
        $A_2$: & \textcolor{blue}{$a_{21}$} & \textcolor{purple}{$a_{22}$} & \textcolor{brown}{$a_{23}$} & \textcolor{green}{...} & $a_{2h}$ & ... \\
        $A_3$: & \textcolor{purple}{$a_{31}$} & \textcolor{brown}{$a_{32}$} & \textcolor{green}{$a_{33}$} & ... & $a_{3h}$ & ... \\
        ... & \textcolor{brown}{...} & \textcolor{green}{...} & ... & ... & ... & ... \\
        $A_i$: & \textcolor{green}{$a_{i1}$} & $a_{i2}$ & $a_{i3}$ & ... & $a_{ih}$ & ... \\
        ... & ... & .. & ... & ... & ... & ... \\
    \end{tabular}
\end{center}

Consideriamo le diagonali \textcolor{red}{$D_1$}, \textcolor{blue}{$D_2$}, ..., \textcolor{green}{$D_k$}, ..., dove: $D_k = \{a_{ij} \; | \; i + j = k + 1\}$. Per dimostrare che $\#(\bigcup_{i \in \mathbb{N}}A_i)$ è \textbf{numerabile}, occorre costruire una applicazione biunivoca, tale che:
\begin{center}
    $h : \bigcup_{i \in \mathbb{N}_n} A_i \rightarrow \mathbb{N}$
\end{center}
Scorrendo ogni diagonale a partire dall'elemento che sta nell'insieme con indice maggiore, incontrerò l'elemento $a_{ij}$ come j-esimo elemento della diagonale a cui esso appartiene, ovvero come j-esimo elemento della diagonale $D_{i+j-1}$. \\
Osservando che $\#D_k = k$ si ha che:
\begin{center}
    $\sum_{k=1}^{i+j-2}\#D_k = \frac{(i + j - 2)(i + j - 1)}{2}$
\end{center}
Si ottiene così una applicazione biunivoca $h : \bigcup_{i \in \mathbb{N}_n}A_i \rightarrow \mathbb{N}$ definita, $\forall a_{ij} \in \bigcup_{i \in \mathbb{N}_n} A_i$, da:
\begin{center}
    $h(a_{ij}) = j + \frac{(i + j - 2)(i + j - 1)}{2}$
\end{center}

\textbf{Conseguenze}:
\begin{itemize}
    \item $\mathbb{Z}$ è numerabile: $\mathbb{Z} = \mathbb{N} \cup \{0\} \cup \{ -n \; | \; n \in \mathbb{N}\}$.
    \item $\mathbb{N} \times \mathbb{N}$ è numerabile: $\mathbb{N} \times \mathbb{N} = \{(n, m) \; | \; n, m \in \mathbb{N}\} = \bigcup_{n \in \mathbb{N}} \{(n, m) \; | \; m \in \mathbb{N}\}$.
    \item $\mathbb{Q}$ è numerabile.
\end{itemize}

\section{Confronto tra Cardinalità}
Si dice che un insieme A ha \textbf{cardinalità minore o uguale} ad un insieme B (e si indica con: $\#A <=\#B$) se: $\exists f : A \rightarrow B, \; f \; è \; iniettiva$. \\
\textbf{Proprietà}:
\begin{itemize}
    \item \textbf{riflessività}: $\forall A, \; \#A \leq \#A$.
    \item \textbf{transitività}: $\#A \leq \#B, \; \#B \leq \#C \Rightarrow \#A \leq \#C$.
    \item \textbf{antisimmetria}: $\#A \leq \#B, \; \#B \leq \#A \Rightarrow \#A = \#B$.
    \item \textbf{tricotomia}: $\forall A, B \Rightarrow \#A \leq \#B \; o \; \#B \leq \#A$.
\end{itemize}
La relazione ``$\leq$'' fra cardinalità è una relazione di ordine totale. \\
$A \subseteq B \subseteq C$ con $\#A = \#B \Rightarrow \#A = \#B = \#C$. \\
\newline
\textbf{Teorema di Cantor-Bernstein-Schroeder}: Se $\exists f : A \rightarrow B, \; f \; iniettiva$ ed $\exists g : B \rightarrow A, \; g \; iniettiva$ allora $\exists h : A \rightarrow B, \; h \; biunivoca$. \\
\textcolor{green}{\textbf{Dimostrazione}}