%\documentclass[12pt,a4paper]{report}
\documentclass[12pt,a4paper,oneside,onecolumn,openright]{book}
% set the document language
\usepackage[italian]{babel}
% set the encoding used by your editor here (default is utf8)
\usepackage[utf8]{inputenc}
\usepackage[T1]{fontenc}

% math packages
\usepackage{amsmath}
\usepackage{amssymb}
\usepackage{lmodern}
\usepackage{makecell}
\usepackage[table]{xcolor}
\usepackage{colortbl}

% page margins settings
\usepackage[inner=3cm,outer=2.5cm,top=3cm,bottom=2.5cm]{geometry}
%\usepackage{indentfirst}

% other packages
\usepackage{array}
\usepackage{enumitem}
\usepackage{subfigure}
\usepackage{graphicx}
\usepackage{verbatim}
\usepackage{listings}
\usepackage{url}
\usepackage[hidelinks]{hyperref}
\usepackage[export]{adjustbox}
\usepackage{latexsym}
\usepackage{tabularx}
\usepackage{ragged2e}
\usepackage{mathtools}
\DeclarePairedDelimiter\floor{\lfloor}{\rfloor}
\DeclarePairedDelimiter\ceil{\lceil}{\rceil}
% \usepackage{Mathematics}
% custom colors
\usepackage{color}
\usepackage{wrapfig}
\usepackage{gensymb}
\usepackage{caption}
\usepackage{tikz}
\usepackage{forest}
\usepackage{tikz-qtree}
\usetikzlibrary{positioning, shapes.geometric}
\usetikzlibrary{shapes.geometric, arrows.meta, positioning, fit}
\tikzstyle{block} = [thick, text width=.4cm, minimum height=.5cm, align=center] 


\usetikzlibrary{shadows}
\definecolor{light-gray}{gray}{0.96}
\definecolor{cyan}{RGB}{230,230,255}
\definecolor{dkgreen}{rgb}{0,0.6,0}
\definecolor{gray}{rgb}{0.5,0.5,0.5}
\definecolor{mauve}{rgb}{0.58,0,0.82}
\definecolor{iceberg}{rgb}{0.44, 0.65, 0.82}
% \definecolor{blue}{RGB}{44, 44, 210}

\hypersetup{
colorlinks=true,
linkcolor=black,
% filecolor=blue,
urlcolor=blue,
% pdftitle={Overleaf Example},
}

\urlstyle{same}
\graphicspath{ {./images/} }

\usepackage[many]{tcolorbox}
\newtcolorbox{boxA}{
    % fontupper = \bf,
    boxrule = 1.5pt,
    colframe = black % frame color
}

% environment for bash code
\lstset{ %
  language=bash,                % the language of the code
  basicstyle=\footnotesize,           % the size of the fonts that are used for the code
  numbers=left,                   % where to put the line-numbers
  numberstyle=\footnotesize,          % the size of the fonts that are used for the line-numbers
  stepnumber=1,                   % the step between two line-numbers. If it's 1, each line 
                                  % will be numbered
  numbersep=5pt,                  % how far the line-numbers are from the code
  backgroundcolor=\color{white},      % choose the background color. You must add \usepackage{color}
  showspaces=false,               % show spaces adding particular underscores
  showstringspaces=false,         % underline spaces within strings
  showtabs=false,                 % show tabs within strings adding particular underscores
%  frame=single,                   % adds a frame around the code
  rulecolor=\color{black},        % if not set, the frame-color may be changed on line-breaks within not-black text (e.g. commens (green here))
  tabsize=2,                      % sets default tabsize to 2 spaces
  captionpos=b,                   % sets the caption-position to bottom
  breaklines=true,                % sets automatic line breaking
  breakatwhitespace=false,        % sets if automatic breaks should only happen at whitespace
  title=\lstname,                   % show the filename of files included with \lstinputlisting;
                                  % also try caption instead of title
  numberstyle=\tiny\color{gray},        % line number style
  keywordstyle=\textbf,          % keyword style
  commentstyle=\color{dkgreen},       % comment style
%  stringstyle=\color{mauve},         % string literal style
  escapeinside={\%*}{*)},            % if you want to add a comment within your code
  morekeywords={*,...,insert,-}               % if you want to add more keywords to the setù
}

% environment for python code
\lstset{
	language=Python,
	breaklines=true,
	breakatwhitespace=true ,
	backgroundcolor=\color{light-gray}
}


\newcommand{\grayScale}{0.95} % Can change the gray level here
\definecolor{codeBackground}{rgb}{\grayScale ,\grayScale ,\grayScale}
\definecolor{forestGreen}{rgb}{0.13,0.55,0.13}

\lstset{
    language=C,
    backgroundcolor=\color{codeBackground},
    tabsize=4,
    showstringspaces=false,
    showtabs=false,
    showspaces=false,
    basicstyle=\ttfamily,
    identifierstyle=\ttfamily,
    keywordstyle=\color{blue},
    stringstyle=\color{red},
    commentstyle=\color{gray},
    numberstyle=\color{magenta},
    morecomment=[l][\color{forestGreen}]{\#},
    escapechar={|}, 
}
% appendices package
%\usepackage{appendix}
% set Appendix name used in the toc
%\renewcommand{\appendixtocname}{Appendice}

% interline
\linespread{1.5}
% set numbers for subsections and show them in the toc
\setcounter{tocdepth}{3} 
\setcounter{secnumdepth}{3}

% layout package, style and settings
\usepackage{fancyhdr}
\pagestyle{fancy}

\fancypagestyle{mainmatter}{%		
		\fancyhf{} 
		\fancyhead{}
		\fancyhead[LE,RO]{\thepage}
		\fancyhead[LO]{\footnotesize{\leftmark}}
		\fancyhead[RE]{\footnotesize{\rightmark}}
		\fancyfoot{}
		\addtolength{\headwidth}{\marginparsep}
		\addtolength{\headheight}{2.5pt}
		\renewcommand{\headrulewidth}{0.3pt}
		\renewcommand{\footrulewidth}{0.0pt}
		}
\fancypagestyle{frontmatter}{%
		\fancyhf{} 
		\fancyhead[LE]{\footnotesize{\MakeUppercase{\thepage}}}
		\fancyhead[RO]{\footnotesize{\MakeUppercase{\thepage}}}
		\fancyhead[RE,LO]{}
		\fancyfoot{}
		\addtolength{\headwidth}{\marginparsep}
		\addtolength{\headheight}{2.5pt}
		\renewcommand{\headrulewidth}{0.0pt}
		\renewcommand{\footrulewidth}{0.0pt}
		}
		
		
\usepackage{fancyhdr}
\pagestyle{fancy}
		\fancyhf{} 
		\fancyhead{}
		\fancyhead[LE,RO]{\thepage} 
		\fancyhead[LO]{\footnotesize{\leftmark}}
		\fancyhead[RE]{\footnotesize{\rightmark}}
		\fancyfoot{}
		\addtolength{\headwidth}{\marginparsep}
		\addtolength{\headheight}{2.5pt}
		\renewcommand{\headrulewidth}{0.3pt}
		\renewcommand{\footrulewidth}{0.0pt}

% empty pages have no numbers
\makeatletter
\def\cleardoublepage{\clearpage\if@twoside \ifodd\c@page\else
\hbox{}
  %Potresti voler togliere il commento dalla linea seguente
  %Questa pagina � stata lasciata intenzionalmente vuota.
\thispagestyle{empty}
\newpage
\if@twocolumn\hbox{}\newpage\fi\fi\fi}
\makeatother
%????
%\textwidth=450pt\oddsidemargin=0pt

%\makeatletter 
%  \DeclareRobustCommand*\textsubscript[1]{% 
%    \@textsubscript{\selectfont#1}} 
%  \newcommand{\@textsubscript}[1]{% 
%    {\m@th\ensuremath{_{\mbox{\fontsize\sf@size\z@#1}}}}} 
\makeatother 

\begin{document}
\begin{titlepage}
\begin{center}
{
    \large
    \textbf{Università  degli studi di Modena e Reggio Emilia} \\
   	\textbf{Dipartimento di Ingegneria Enzo Ferrari} \\
    \vspace{\stretch{0.5}}
    \hspace*{0cm} \hrulefill \hspace*{0cm} \\
    \vspace{\stretch{0.5}}    
	  \vspace{\stretch{12}}
  
  
 		\huge{\bf Operating System Design }}\\
		\vspace{3mm}
		
		\vspace{\stretch{6}}
		\end{center}
		
\vspace{40mm}
\par
\noindent
\vspace{20mm}
\begin{center}
\hspace*{0cm} \hrulefill \hspace*{0cm} \\
{\large{\bf 
Anno Accademico 2024/25}}
\end{center}

\end{titlepage}

\pagestyle{frontmatter}
\frontmatter

% PAGINA VUOTA
%\clearpage\null\thispagestyle{empty}\clearpage
\setcounter{tocdepth}{2}
\tableofcontents

\setlength{\parindent}{12pt}
\setlength{\parskip}{1ex plus 0.5ex minus 0.2ex}
\mainmatter
\pagestyle{mainmatter}

\chapter{Processi e Thread}

\section{Processi}

\textbf{Processo}: è l'\textit{astrazione} di un programma in esecuzione. Il processo è l'astrazione più elementare e più importante che ci può fornire il sistema operativo. Riuscendo ad emulare il comportamento di esecuzione concorrente nonostante la presenza di una singola CPUs. \\
Un altro paio di definizioni:
\begin{itemize}[nosep]
    \item \textbf{algoritmo}: in matematica ed informatica ci si riferisce ad algoritmo ad una sequenza finita di istruzioni rigorose matematiche che hanno l'obiettivo di risolvere una determinata classe di un problema specifico o di risolvere un calcolo.
    \item \textbf{programma}: è la sequenza o l'insieme di istruzioni in liguaggio macchina che può essere eseguito.
\end{itemize}
I moderni computer possono eseguire diverse operazioni nello stesso istante. Descrivendo in maniera rigida quello che effettivamente succede, però, è che ogni CPU in \textit{ogni istante di tempo} esegue \textbf{uno e un solo} processo. Tendendo a 0 il tempo riservato a ogni singolo processo è però possibile simulare \textbf{parallelismo} definito anche come: \textbf{\textit{pseudoparallelism}}, che però va in contrasto con il vero parallelismo hardware (multi-CPUs). \\
Il \textbf{\textit{Process Model}} definisce che tutti gli eseguibili del computer, a volte includendo il sistema operativo, vengano organizzati in una serie di \textbf{processi sequenziali}. Il processo è stato definito come l'istanza di un programma in esecuizione nel quale viene anche incluso il suo \textbf{\textit{PCB}} (\textbf{\textit{Process Control Block}}). Il \textbf{PCB}, anche noto come \textit{process descriptor} è una struttura che permette di salvare tutte le informazioni che riguardano un determinato processo, ad esempio: \textit{program counter}, i registri e le variabili. Concettualmente possiamo visualizzare che ad ogni processo è associata una CPU virtuale.
\begin{boxA}
    \textcolor{red}{\textbf{\textit{Process Switching}}} \\
    È quando l'\textit{OS} cambia processo in esecuzione sulla CPU. 
\end{boxA}
Per ora considerermo che esista un'\textbf{unica CPU}. Questa assunzione non tiene normalmente conto dei moderni \textit{chip} che sono spesso multi-core. \\
Possiamo visualizzare inizialmente il processo come una tupla che contiene: il programma, degli input, degli output e uno \textbf{stato}. Un singolo processore può essere condiviso da $n$ processi con un algoritmo di \textit{scheduling} (\textit{scheduler algorithm}) che viene utilizzato per determinare quando interrompere un processo (se può farlo) e servirne un altro. \\
\begin{boxA}
    \textcolor{blue}{\textbf{Processo vs. Programma}} \\
    Un programma è qualcosa che può essere salvato su disco, statico; mentre un processo è qualcosa di dinamico e che varia ad ogni sua istanza. \\
    Un programma può essere eseguito da più processi che però sono distinti l'uno dall'altro.
\end{boxA}
La \textbf{creazione di un processo} può essere indotta da:
\begin{itemize}[nosep]
    \item inizializzazione di sistema
    \item un processo in esecuzione compie una \textit{system call} che inizializza un nuovo processo
    \item un utente richiede l'esecuzione di un nuovo processo
\end{itemize}
I processi possono essere eseguiti in \textit{foreground}, ovvero con i quali un utente può interagire, oppure in \textit{background}, che sono ``nascosti'' all'utente e rispondono a certe specifiche funzioni. Su linux sono presenti decina di processi in background, alcuni anche noti come \textit{daemons}.
\begin{boxA}
    In \textbf{UNIX} è presente una solo \textit{system call} per creare un nuovo processo: \textbf{fork}. Dopo l'esecuzione della \textit{syscall} i due processi, il padre e il figlio, hanno la stessa immagine della memoria, le stesse stringhe di environment e gli stessi file aperti. Normalmente, dopo il figlio, esegue \textbf{execve} o una \textit{system call} simile per cambiare l'area di memoria ed eseguire un nuovo programma. \\
    Alcuni implementazioni di \textbf{UNIX} condividono la sezione \textit{.text} tra i due visto che non può essere modificata. In alternativa altre implementazioni possono condividere tutta la memoria del padre, in questo caso la memoria è condivisa in maniera \textbf{\textit{copy-on-write}}, ovvero ogni volta che uno dei due vuole modificare parte della memoria, quel specifico \textit{chunck} viene copiato prima della modifica in una locazione privata della memoria
\end{boxA}



% PAGINA VUOTA
%\clearpage\null\thispagestyle{empty}\clearpage
%\appendix
%\appendixpage
%\addappheadtotoc

%\clearpage\null\thispagestyle{empty}\clearpage


%\listoffigures


\begin{flushleft}
\bibliographystyle{plain}
\bibliography{sections/references} 
\end{flushleft}

\end{document}
